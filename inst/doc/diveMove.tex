%\VignetteIndexEntry{Dive Analysis}
%\VignetteDepends{diveMove}
%\VignettePackage{diveMove}
\documentclass[12pt, letterpaper]{scrartcl}
\usepackage[utf8]{inputenc}
\usepackage[T1]{fontenc}
\usepackage{lmodern}
\usepackage{pdfdraftcopy}
\usepackage[margin=3cm]{geometry}
\usepackage{graphicx}
\usepackage{paralist}
\usepackage{titletoc}
\usepackage{booktabs}
\usepackage{fancyhdr}
\usepackage{natbib}
\usepackage{pdfpages}

\usepackage[colorlinks=true]{hyperref}

\draftstring{DRAFT}
\newcommand{\Robject}[1]{{\texttt{#1}}}
\newcommand{\Rfunction}[1]{{\texttt{#1}}}
\newcommand{\Rpackage}[1]{{\texttt{#1}}}
\newcommand{\Rclass}[1]{{\textit{#1}}}
\newcommand{\Rmethod}[1]{{\textit{#1}}}
\newcommand{\Rfunarg}[1]{{\textit{#1}}}
\newcommand{\R}{{\normalfont\textsf{R }}{}}
\pagestyle{fancy}
\fancyhf{}
% \fancyheadoffset[LO,RE{1em}
% \fancyhead[LE,RO]{\nouppercase \rightmark}
\fancyhead[LE,RO]{\bfseries \nouppercase \leftmark}
\fancyfoot[C]{\thepage}
\setkomafont{captionlabel}{\sffamily\bfseries}
\renewcommand{\captionformat}{.\ }
\bibpunct{(}{)}{,}{a}{}{,}

\setlength{\parindent}{0em}
\setlength{\parskip}{0.5em}
%%%%%%%%%%%%%%%%%%%%%%%%%%%%%%%%%%%%%%%%%%%%%%%%%%%%%%%%%%%%%%%%%%%%%%%%%%%

\usepackage{/usr/share/R/share/texmf/Sweave}
\begin{document}

\title{\bfseries{diveMove: dive analysis in R}}
\author{Sebastián P. Luque\thanks{Contact: \texttt{spluque@gmail.com}.
    Comments for improvement are very welcome!}}
\date{}

\maketitle


\tableofcontents

\section{Introduction}

Dive analysis usually involves handling of large amounts of data, as new
instruments allow for frequent sampling of variables over long periods of
time.  The aim of this package is to make this process more efficient for
summarizing and extracting information gathered by time-depth recorders
(TDRs, hereafter).  The principal motivation for developing
\Rpackage{diveMove} was to provide more flexibility during the various
stages of analysis than that offered by popular commercial software.  This
is achieved by making the results from intermediate calculations easily
accessible, allowing the user to make his/her own summaries beyond the
default choices the package provides.  The following sections of this
vignette illustrate a typical work flow during analysis of TDR data, using
the \Robject{dives} data available in \Rpackage{diveMove} as an example.


\section{Starting up}
\label{sec:starting}

As with other packages in R, to use the package we load it with the
funtion \Rfunction{library}:
\begin{Schunk}
\begin{Sinput}
> library(diveMove)
\end{Sinput}
\end{Schunk}

This makes the objects in the package available in the current \R session.
A short overview of the most important functions can be seen by running
the examples in the package's help page:
\begin{Schunk}
\begin{Sinput}
> example(diveMove)
\end{Sinput}
\end{Schunk}

\section{Reading Input Files}
\label{sec:read-in}

Input files must be simple, comma-delimited text files\footnote{The
  extension does not matter, but conventionally these files have a .csv
  extension}.  The order of columns is not significant, as the column
numbers indicating the variables of interest can be supplied as arguments.
Table~\ref{tab:file-str} shows the file structure that readTDR assumes by
default, which is a standard structure of files from common TDR models.

\begin{table}[bt]
  \caption{Sample TDR file structure.}
  \centering
  \begin{tabular}{rrrrrr}\toprule
    date       & time     & depth  & light  & temperature & speed    \\\midrule
    16/02/2004 & 14:30:00 & 12     & 200    & 8.4         & 1.44     \\
    16/02/2004 & 14:30:05 & 15     & 180    & 8.0         & 1.75     \\
    16/02/2004 & 14:30:10 & 19     & 170    & 7.6         & 1.99     \\
    \ldots     & \ldots   & \ldots & \ldots & \ldots      & \ldots   \\\bottomrule
  \end{tabular}
  \label{tab:file-str}
\end{table}

Depending on the TDR model, speed may be omitted.

To read the file into \R{}, use the function \Rfunction{readTDR}:
\begin{Schunk}
\begin{Sinput}
> sealX <- readTDR(system.file(file.path("data", 
+     "dives.csv"), package = "diveMove"), speed = TRUE)
\end{Sinput}
\end{Schunk}
Read the help page for \Rfunction{readTDR} using \verb|?readTDR| following
common \R{} help facilities.  Thus, data could have been subsampled at a
larger interval than that in the original file, so that the time interval
between readings is 10 s:
\begin{Schunk}
\begin{Sinput}
> sealX <- readTDR(system.file(file.path("data", 
+     "dives.csv"), package = "diveMove"), speed = TRUE, 
+     subsamp = 10)
\end{Sinput}
\end{Schunk}
But since the original 5 s interval (which is the default value for
\Rfunarg{subsamp}) is what will be used for the subsequent sections, it is
recreated it here:
\begin{Schunk}
\begin{Sinput}
> sealX <- readTDR(system.file(file.path("data", 
+     "dives.csv"), package = "diveMove"), speed = TRUE)
\end{Sinput}
\end{Schunk}

The format in which date and time should be interpreted can be controlled
with the argument \Rfunarg{dtformat}.  If the data are already available
in the \R{} session, e.g. as a \Robject{data frame}, then the function
\Rfunction{createTDR} can be used to convert it to a form that facilitates
further analyses.

Both of these functions store the data in an object of class \Rclass{TDR}
or \Rclass{TDRspeed}, which hold information on the source file and
sampling interval, in addition to the variables described above.  Which of
these objects is created is determined by the \Rfunarg{speed}.


\section{Extraction and Display of Information from TDR and TDRspeed
  Objects}
\label{sec:extract}

For convenience, extractor methods are available to access the different
slots from objects of these classes.  The standard \Rmethod{show} method
will display the usual overview information on the object:
\begin{Schunk}
\begin{Sinput}
> sealX
\end{Sinput}
\begin{Soutput}
Time-Depth Recorder data -- Class TDRspeed object
  Source File             : dives.csv 
  Sampling Interval (s)   : 5 
  Number of Samples       : 34199 
  Sampling Begins         : 2002-01-05 11:32:00 
  Sampling Ends           : 2002-01-07 11:01:50 
  Total Duration (d)      : 1.979 
  Measured depth range (m): [ -4 , 91 ]
  Other variables         : light temperature speed 
\end{Soutput}
\end{Schunk}

Other extractor methods are named after the component they extract:
\Rmethod{getTime}, \Rmethod{getDepth}, \Rmethod{getSpeed}, and
\Rmethod{getDtime}, where the latter extracts the sampling interval.  The
\Rmethod{plotTDR} method brings up a plot of the data covering the entire
record, although a \Rpackage{tcltk} widget provides controls for zooming
and panning to any particular time window.  Alernatively, the underlying
function \Rfunction{plotTD} provides the same functionality, but takes
separate \Rfunarg{time} and \Rfunarg{depth} arguments, rather than a
\Rclass{TDR} object.

At any time, TDR objects can be coerced to a simple data frame, which can
later be exported or manipulated any other way:
\begin{Schunk}
\begin{Sinput}
> sealX.df <- as.data.frame(sealX)
> head(sealX.df)
\end{Sinput}
\begin{Soutput}
                 time depth light temperature speed
1 2002-01-05 11:32:00    NA    NA          NA    NA
2 2002-01-05 11:32:05    NA    NA          NA    NA
3 2002-01-05 11:32:10    NA    NA          NA    NA
4 2002-01-05 11:32:15    NA    NA          NA    NA
5 2002-01-05 11:32:20    NA    NA          NA    NA
6 2002-01-05 11:32:25    NA    NA          NA    NA
\end{Soutput}
\end{Schunk}


\section[ZOC and Wet/Dry period detection]{Zero-Offset Depth Correction
  and Summary of Wet/Dry Periods}
\label{sec:zoc}

One the first steps of dive analysis involves correcting depth for shifts
in the pressure transducer, so that surface readings correspond to the
value zero.  Although some complex algorithms exist for detecting where
these shifts occur in the record, the shifts remain difficult to detect
and dives are often missed, which a visual examination of the data would
have exposed.  The trade off is that visually zero-adjusting depth is
tedious, but the advantages of this approach far outweigh this cost, as
much insight is gained by visually exploring the data.  Not to mention the
fact that obvious problems in the records are more effectively dealt with
in this manner.

That personal rant aside, zero offset correction (ZOC) is done in
\Rpackage{diveMove} using the function \Rfunction{zoc}.  However, a more
efficient method of doing this is by using the \Rfunction{calibrateDepth}
function, which takes a \Rclass{TDR} object (or inheriting from it) to
perform three basic tasks.  The first is to ZOC the data, using the
\Rpackage{tcltk} package to be able to do it interactively:
\begin{Schunk}
\begin{Sinput}
> dcalib <- calibrateDepth(sealX)
\end{Sinput}
\end{Schunk}
% The code we actually run is below, to be able to provide own value

This command brings up a plot with \Rpackage{tcltk} controls allowing to
pan and zoom in or out of the data, as well as adjustment of the
\Robject{depth} scale.  Thus, an appropriate time window with a unique
surface depth value can be displayed.  It is important to make the display
such that the \Robject{depth} scale is small enough to allow the
resolution of the surface value with the mouse.  Clicking on the ZOC
button waits for two clicks:
\begin{compactenum}
\item the coordinates of the first click define the starting time for the
  window to be ZOC'ed, and the depth corresponding to the surface,
\item the second click defines the end time for the window (only the x
  coordinate has any meaning).
\end{compactenum}
This procedure can be repeated as many times as needed.  If there is any
overlap between time windows, then the last one prevails.  However, if the
offset is known a priori, there is no need to go through all this
procedure, and the value can be provided as the argument \Rfunarg{offset}
to \Rfunction{calibrateDepth}.

Once depth has been ZOC'ed, \Rfunction{calibrateDepth} will identify dry
and wet periods in the record.  Wet periods are those where a depth
reading is available, dry periods are those without a depth reading.
Records often have abherrant missing depth that should not be considered
dry periods, as they are often of very short duration.  Likewise, there
may be periods of wet activity that are too short to be compared with
other wet periods.  This can be controlled by setting the arguments
\Rfunarg{dry.thr} and \Rfunarg{wet.thr}.

Finally, \Rfunction{calibrateDepth} identifies all dives in the record,
according to a minimum depth criteria given as its \Rfunarg{divethres}
argument.  The result (value) of this function is an object of class
\Rclass{TDRcalibrate}, where all the information obtained during the tasks
described above are stored.  Again, an appropriate \Rmethod{show} method
is available to display a short overview of such objects:
\begin{Schunk}
\begin{Sinput}
> dcalib
\end{Sinput}
\begin{Soutput}
Depth calibration -- Class TDRcalibrate object
  Source file                       : dives.csv 
  Number of dry phases              : 4 
  Number of aquatic phases          : 3 
  Number of dives detected          : 317 
  Dry threshold used (s)            : 70 
  Aquatic theshold used (s)         : 3610 
  Dive threshold used (s)           : 4 
  Speed calibration coefficients    : a = 0 ; b = 1 
\end{Soutput}
\end{Schunk}


\section{Access to Elements from \Rclass{TDRcalibrate} Objects}
\label{sec:wet-dry}

Extractor methods are also available to access the information stored in
\Rclass{TDRcalibrate} objects.  These include: \Rmethod{getTDR},
\Rmethod{getGAct}, \Rmethod{getDAct}, \Rmethod{getDPhaseLab}, and
\Rmethod{getSpeedCoefs}.  These are all generic functions\footnote{A few
  of them with more than one method} that access the (depth) calibrated
\Rclass{TDR} object, details from wet/dry periods, dives, dive phases, and
speed calibration coefficients (see~Section~\ref{sec:speed}),
respectively.  Below is a short explanation of these methods.

\begin{itemize}
\item [\Rmethod{getTDR}] This method simply takes the
  \Rclass{TDRcalibrate} object as its single argument and extracts the
  \Rclass{TDR} object\footnote{In fact, a \Rclass{TDRspeed} object in this
    example}:
\begin{Schunk}
\begin{Sinput}
> getTDR(dcalib)
\end{Sinput}
\begin{Soutput}
Time-Depth Recorder data -- Class TDRspeed object
  Source File             : dives.csv 
  Sampling Interval (s)   : 5 
  Number of Samples       : 34199 
  Sampling Begins         : 2002-01-05 11:32:00 
  Sampling Ends           : 2002-01-07 11:01:50 
  Total Duration (d)      : 1.979 
  Measured depth range (m): [ 0 , 88 ]
  Other variables         : light temperature speed 
\end{Soutput}
\end{Schunk}

\item [\Rmethod{getGAct}] There are two methods for this generic, allowing
  access to a list with details about all wet/dry periods found.  One of
  these extracts the entire \Rclass{list} (output omitted for brevity):
\begin{Schunk}
\begin{Sinput}
> getGAct(dcalib)
\end{Sinput}
\end{Schunk}
The other provides access to particular elements of the \Rclass{list}, by
their name.  For example, if we are interested in extracting only the
vector that tells us to which period number every row in the record
belongs to, we would issue the command:
\begin{Schunk}
\begin{Sinput}
> getGAct(dcalib, "phase.id")
\end{Sinput}
\end{Schunk}
Other elements that can be extracted are named ``activity'', ``begin'',
and ``end'', and can be extracted in a similar fashion.  These elements
correspond to the activity performed for each reading (see
\verb|?detPhase| for a description of the labels for each activity), the
beginning and ending time for each period, respectively.

\item [\Rmethod{getDAct}] This generic also has two methods; one to
  extract an entire data frame with details about all dive and postdive
  periods found (output omitted):
\begin{Schunk}
\begin{Sinput}
> getDAct(dcalib)
\end{Sinput}
\end{Schunk}
The other method provides access to the columns of this data frame, which
are named ``dive.id'', ``dive.activity'', and ``postdive.id''.  Thus,
providing any one of these strings to getDAct, as a second argument will
extract the corresponding column.

\item [\Rmethod{getDPhaseLab}] This generic function extracts a factor
  identifying each row of the record to a particular dive phase (see
  \verb|?detDive| for a description of the labels of the factor
  identifying each dive phase).  Two methods are available; one to extract
  the entire factor, and the other to select particular dive(s), by its
  (their) number, respectively (output omitted):
\begin{Schunk}
\begin{Sinput}
> getDPhaseLab(dcalib)
> getDPhaseLab(dcalib, 20)
\end{Sinput}
\end{Schunk}
\begin{Schunk}
\begin{Sinput}
> dphases <- getDPhaseLab(dcalib, c(100:300))
\end{Sinput}
\end{Schunk}
The latter method is useful for visually inspecting the assignment of
points to particular dive phases.  Before doing that though, this is a
good time to introduce another generic function that allows the subsetting
of the original \Rclass{TDR} object to a single a dive or group of dives'
data:
\begin{Schunk}
\begin{Sinput}
> subSealX <- extractDive(dcalib, diveNo = c(100:300))
> subSealX
\end{Sinput}
\begin{Soutput}
Time-Depth Recorder data -- Class TDRspeed object
  Source File             : dives.csv 
  Sampling Interval (s)   : 5 
  Number of Samples       : 2410 
  Sampling Begins         : 2002-01-06 00:45:15 
  Sampling Ends           : 2002-01-07 03:27:10 
  Total Duration (d)      : 1.112 
  Measured depth range (m): [ 0 , 88 ]
  Other variables         : light temperature speed 
\end{Soutput}
\end{Schunk}

As can be seen, the function takes a \Rclass{TDRcalibrate} object and a
vector indicating the dive numbers to extract, and returns a \Rclass{TDR}
object containing the subsetted data.  Once a subset of data has been
selected, it is possible to plot them and pass the factor labelling dive
phases as the argument \Rfunarg{phaseCol} to the \Rmethod{plotTDR}
method\footnote{The function that the method uses is actually
  \Rfunction{plotTD}, so all the possible arguments can be studied by
  reading the help page for \Rfunction{plotTD}}:

\begin{Schunk}
\begin{Sinput}
> plotTDR(subSealX, phaseCol = dphases)
\end{Sinput}
\end{Schunk}

\end{itemize}


\section{Speed Calibration}
\label{sec:speed}

Calibration of speed readings is done using the principles described in
\citet{1929} and \citet{1291}.  The function \Rfunction{calibrateSpeed}
performs this operation\footnote{CAUTION: This implementation is
  experimental, and may give unexpected results.}, and allows the
selection of the particular subset of the data that should be used for the
calibration:

\begin{Schunk}
\begin{Sinput}
> vcalib <- calibrateSpeed(dcalib, z = 1)
\end{Sinput}
\end{Schunk}
\begin{Schunk}
\begin{Sinput}
> vcalib
\end{Sinput}
\begin{Soutput}
Depth calibration -- Class TDRcalibrate object
  Source file                       : dives.csv 
  Number of dry phases              : 4 
  Number of aquatic phases          : 3 
  Number of dives detected          : 317 
  Dry threshold used (s)            : 70 
  Aquatic theshold used (s)         : 3610 
  Dive threshold used (s)           : 4 
  Speed calibration coefficients    : a = -0.44 ; b = 1.1 
\end{Soutput}
\end{Schunk}
\begin{figure}[hbtp]
  \centering
  \includegraphics[trim=0 0 10 0, width=0.5\textwidth]{diveMove-vcalibration}
  \caption{Example speed calibration line from a TDR record.}
  \label{fig:rqplots}
\end{figure}

Using \Rfunarg{plot=FALSE} it is possible to turn off the default side
effect of producing a plot displaying the quantile regression fit
(Figure~\ref{fig:rqplots}).

Control is possible by the use of arguments \Rfunarg{bad}, which controls
minimum rates of depth change and speeds through which the calibration
line should be drawn.  To control for the resolution of the TDR,
\Rfunarg{z} can be used to include only changes in depth greater than a
given value for the construction of the calibration line.

If the calibration coefficients from the implicit quantile regression are
known a priori, then these can be supplied to the function via its
\Rfunarg{coefs} argument.  In this case, no plots are created.


\section{TDR dive and postdive statistics}
\label{sec:tdr-stats}

Once data have been calibrated and the record broken up at ``trip'' and
``dive'' scales, obtaining dive statistics is a trivial call to function
\Rfunction{diveStats}:

\begin{Schunk}
\begin{Sinput}
> dives <- diveStats(vcalib)
> head(dives, 3)
\end{Sinput}
\begin{Soutput}
              begdesc             enddesc
1 2002-01-05 12:20:10 2002-01-05 12:20:10
2 2002-01-05 21:19:40 2002-01-05 21:20:10
3 2002-01-05 21:22:10 2002-01-05 21:23:05
               begasc desctim botttim asctim descdist
1 2002-01-05 12:20:25     2.5      15    2.5        3
2 2002-01-05 21:20:50    32.5      40   37.5       24
3 2002-01-05 21:23:50    57.5      45   72.5       61
  bottdist ascdist desc.tdist desc.mean.speed desc.angle
1        6       3         NA              NA         NA
2        9      25      63.62           2.121      22.16
3       10      67      98.08           1.783      38.46
  bott.tdist bott.mean.speed asc.tdist asc.mean.speed
1      42.87           2.858        NA             NA
2      87.59           2.190     55.67          1.591
3      69.92           1.554    108.13          1.545
  asc.angle divetim maxdep postdive.dur postdive.tdist
1        NA      20      6        32345       52784.67
2     26.69     110     29           35          35.78
3     38.29     175     67           75          89.21
  postdive.mean.speed
1               1.638
2               1.022
3               1.189
\end{Soutput}
\end{Schunk}

The function takes a single argument: an object of class
\Rclass{TDRcalibrate}, and returns a data frame with one row per dive in
the record, with a suite of basic dive statistics in each column.  Please
consult \verb|?diveStats| for an explanation of each of the variables
estimated, although the names of the output data frame should be self
explanatory.  These variables are thus available for calculating any other
derived values, by extracting them using the standard \R{} subscripting
facilities.


\section{Miscellaneous functions}
\label{sec:misc}

Other functions are included for handling location data, and these are
\Rfunction{readLocs}, \Rfunction{austFilter}, and \Rfunction{distSpeed}.
These are useful for reading, filtering, and summarizing travel
information.  For extensive animal movement analyses, refer to package
\Rpackage{trip}.


\section{Acknowledgements}
\label{sec:acknowledgements}

Invaluable input and help during development of this package has been
offered by John~P.Y. Arnould, and regular contributors to R-help.


\bibliographystyle{apalike}
\bibliography{/home/sluque/Documents/References/biblioSPL}


\newpage{}
\includepdf[fitpaper=true, pages=-]{diveMove-rd}


\end{document}
